% Define document class. Important.
\documentclass[a4paper,twoside,openright]{memoir}

% simple command to insert our final language name. Use \langname wherever you write the language's name.
\newcommand{\langname}{InvaderScript} % Whitespace intentional.


% Margener
%\setlength{\evensidemargin}{1cm}
%\setlength{\oddsidemargin}{1cm}

% Set up encoding. Latin1 since UTF-8 is fuckably difficult to work with.
% \usepackage[latin1]{inputenc}
\usepackage[latin1,utf8,ansinew]{inputenc}

% Variable, neat, references.
\usepackage{varioref}

% Acronym package. Makes it easier to introduce acronyms.
\usepackage{acronym}
% Add acronyms defined in separate file.
\acrodef{dm}[DM]{Dungeon Master}
\acrodef{rpg}[RPG]{Role-Playing Game}
\acrodef{rpgs}[RPGs]{Role-Playing Games}
\acrodef{vtm}[VTM]{Vampire: The Masquerade}
\acrodef{regex}[Regex]{Regular expressions}
\acrodef{cfg}[CFG]{Context-Free Grammar}
\acrodef{cfgs}[CFGs]{Context-Free Grammars}
\acrodef{bnf}[BNF]{Backus-Naur Form}
\acrodef{ebnf}[EBNF]{Extended Backus-Naur Form}
\acrodef{dnd}[DnD]{Dungeons \& Dragons}
\acrodef{wod}[WoD]{World of Darkness}

\usepackage{latexsym}

%Premable for semantic
%\input{semantic_preamble}

% Load up bibliography.
\usepackage[numbers]{natbib}
% Bibliography style.
\bibliographystyle{plainnat}

% Algorithm support.
\usepackage{algorithmic}
\usepackage{algorithm}
\usepackage{lastpage}
% Make algorithms appear as procedures instead.
\floatname{algorithm}{Procedure}
\renewcommand{\algorithmicrequire}{\textbf{Input:}}
\renewcommand{\algorithmicensure}{\textbf{Output:}}

% Image frames.
\setlength{\fboxsep}{0pt}
\setlength{\fboxrule}{0.5pt}

% Also, images.
\usepackage{graphicx}

% PDF includes
\usepackage{pdfpages}

% Block comments
\usepackage{verbatim}

% Todo notes here and there.
\usepackage{todonotes}

% Forbedrede floats.
\usepackage{float}

% Special symbols availability.
\usepackage{amsmath}
\usepackage{amssymb}
\usepackage{amsthm}

% CODE %
\usepackage{listings}
\usepackage{color}
\definecolor{gray}{rgb}{0.4,0.4,0.4}
\definecolor{darkblue}{rgb}{0.0,0.0,0.6}
\definecolor{cyan}{rgb}{0.0,0.6,0.6}
\lstset{
  basicstyle=\ttfamily,
  columns=fullflexible,
  showstringspaces=false,
  commentstyle=\color{gray}\upshape,
  basicstyle=\small,
  numberstyle=\footnotesize,
  numbers=left,
  captionpos=b,
  stepnumber=1,
  numbersep=10pt,
  tabsize=2,
  breaklines=true,
}
% Define markup of XML
\lstdefinelanguage{XML}
{
  morestring=[b]",
  morestring=[s]{>}{<},
  morecomment=[s]{<?}{?>},
  identifierstyle=\color{darkblue},
  keywordstyle=\color{cyan},
  morekeywords={id, target, type, category, value, point, correct, rows, width, time}% list your attributes here
}
% Define markup of C#
\lstdefinelanguage{CSharp}[Visual]{C++}
{
	identifierstyle=\color{darkblue},
	commentstyle=\color{green!70!black}\itshape ,
	stringstyle=\color{gray},
	sensitive=true,
	morestring=[b]",
	morestring=[b]',
	morecomment=[l]//,
	morecomment=[n]{/*}{*/}
}
% Define markup of JavaScript - copied from:
% http://lenaherrmann.net/2010/05/20/javascript-syntax-highlighting-in-the-latex-listings-package
\lstdefinelanguage{JavaScript}{
  keywords={typeof, new, true, false, catch, function, return, null, catch, switch, var, if, in, while, do, else, case, break},
  keywordstyle=\color{blue}\bfseries,
  ndkeywords={class, export, boolean, throw, implements, import, this},
  ndkeywordstyle=\color{darkgray}\bfseries,
  identifierstyle=\color{black},
  sensitive=false,
  comment=[l]{//},
  morecomment=[s]{/*}{*/},
  commentstyle=\color{purple}\ttfamily,
  stringstyle=\color{red}\ttfamily,
  morestring=[b]',
  morestring=[b]"
}

% External file with definitions of languages.
\definecolor{dkgreen}{rgb}{0,0.6,0}
\definecolor{gray}{rgb}{0.5,0.5,0.5}
\definecolor{mauve}{rgb}{0.58,0,0.82}
\definecolor{orange}{rgb}{1,0.5,0}

\lstdefinestyle{ourstyle}
{
	keywordstyle=\color{blue},          % keyword style
	commentstyle=\color{dkgreen},       % comment style
	stringstyle=\color{mauve},         % string literal style
}

\lstdefinelanguage{ourlang}
{
	keywords = {[1]Character,Ability,Effect,Item,Attribute,Resource,Behaviour,Event,make,from,owner,this,self,enemy},
	morekeywords = {[2]attributes,resources,effects,abilities,items,events,targets,cost,behaviour,team,ally,target, mana_cost, positive, negative,name},
	sensitive = true,
	keywordstyle=\color{blue},          % keyword style
	keywordstyle={[2]\color{dkgreen}},
	commentstyle=\color{dkgreen},       % comment style
	stringstyle=\color{mauve},         % string literal style
}

\lstdefinelanguage{fflang}[]{ourlang}
{
	morekeywords = [20]{
		% Classes
		Warrior,WhiteMage,BlackMage,Monk,Ninja,Master,WhiteWizard,Knight,BlackWizard,
		% Attributes
		strength,intelligence,defense,magic,agility,
		%Resources
		health,mana,PhysicalDamage,Heal},
	keywordstyle={[20]\color{orange}}
}

% Neat-o referencer...o.
\usepackage{hyperref}
\usepackage{nameref}

% hack fra nettet.
% http://tex.stackexchange.com/questions/1230/reference-name-of-description-list-item-in-latex
\makeatletter
\let\orgdescriptionlabel\descriptionlabel
\renewcommand*{\descriptionlabel}[1]{%
  \let\orglabel\label
  \let\label\@gobble
  \phantomsection
  \edef\@currentlabel{#1}%
  %\edef\@currentlabelname{#1}%
%  \let\label\orglabel
  \orgdescriptionlabel{#1}%
}
\makeatother
% Rettehak. Meget lettere end \checkmark
\newcommand{\yes}{\checkmark}

% Let's put in a lot of niceness in the display, yeh?

\usepackage{color,calc,graphicx,soul,fourier}
\definecolor{nicered}{rgb}{.647,.129,.149}
\makeatletter
\newlength\dlf@normtxtw
\setlength\dlf@normtxtw{\textwidth}
\def\myhelvetfont{\def\sfdefault{mdput}}
\newsavebox{\feline@chapter}
\newcommand\feline@chapter@marker[1][4cm]{%
\sbox\feline@chapter{%
\resizebox{!}{#1}{\fboxsep=1pt%
\colorbox{nicered}{\color{white}\bfseries\sffamily\thechapter}%
}}%
\rotatebox{90}{%
\resizebox{%
\heightof{\usebox{\feline@chapter}}+\depthof{\usebox{\feline@chapter}}}%
{!}{\scshape\so\@chapapp}}\quad%
\raisebox{\depthof{\usebox{\feline@chapter}}}{\usebox{\feline@chapter}}%
}
\newcommand\feline@chm[1][4cm]{%
\sbox\feline@chapter{\feline@chapter@marker[#1]}%
\makebox[0pt][l]{% aka \rlap
\makebox[1cm][r]{\usebox\feline@chapter}%
}}
\makechapterstyle{daleif1}{
\renewcommand\chapnamefont{\normalfont\Large\scshape\raggedleft\so}
\renewcommand\chaptitlefont{\normalfont\huge\bfseries\scshape\color{nicered}}
\renewcommand\chapternamenum{}
\renewcommand\printchaptername{}
\renewcommand\printchapternum{\null\hfill\feline@chm[2.5cm]\par}
\renewcommand\afterchapternum{\par\vskip\midchapskip}
\renewcommand\printchaptertitle[1]{\chaptitlefont\raggedleft ##1\par}
\setsecindent{2pt}
\setsecheadstyle{\normalfont\Large\bfseries\scshape\color{nicered}}
\setsubsecindent{2pt}
\setsubsecheadstyle{\normalfont\large\bfseries\scshape\color{nicered}}
\setsubsubsecindent{2pt}
\setsubsubsecheadstyle{\normalfont\normalsize\bfseries\scshape\color{nicered}}
}
\makeatother
\chapterstyle{daleif1}


%\usepackage{fancyhdr} % Get some niceness into our headers.
%\pagenumbering{arabic} % Ensure page numbering in our desired form.
%\pagestyle{fancy}
% Page design from fancyhdr.
%\fancyhead{}
%\fancyfoot{}
%\fancyhead[RO,LE]{\leftmark\\\rightmark}
%\fancyfoot[C]{\thepage}
%\setlength{\headheight}{23pt}
% Rewrite header and footer commands.
%\renewcommand{\headrulewidth}{1.0pt}
%\renewcommand{\footrulewidth}{1.0pt}

% Create a new command, HRule, to insert some nice horisontal rules on the title page.
\newcommand{\HRule}{\rule{\linewidth}{0.3mm}}

% New command for two figures, side by side.
\newcommand{\twofigs}[6]
{
	\begin{figure}[H]
		\begin{minipage}[b]{0.5\columnwidth}
		\centering\fbox{\includegraphics[width=0.8\columnwidth]{img/#1}}\caption{#2\label{#3}}
		\end{minipage}
		\hspace{0.5cm}
		\begin{minipage}[b]{0.5\columnwidth}
		\centering\fbox{\includegraphics[width=0.8\columnwidth]{img/#4}}\caption{#5\label{#6}}
		\end{minipage}	\end{figure}
}

% Sørg for at paragrafplads ikke spildes.
\raggedbottom
