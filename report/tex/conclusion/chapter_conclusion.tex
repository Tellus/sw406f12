\chapter{Conclusion}
Roleplaying has been around since the 1970'es, and today many people play \ac{rpgs} both on computer and in the real world in the form of pen and paper and live roleplaying.

Characters in \ac{rpg} systems are described almost entirely in numbers, called stats, and these help both players and \ac{gm} determine the outcome of actions and events in the scenario. 

Combat is often an integral part of the scenarios, but there is no programming language designed to allow description of combat scenarios by inexperienced programmers.

The language \langname{} is a language following the declarative paradigm which allows users to define characters, attributes, resources, abilities, behaviours and events and simulate a combat situation with the defined characters with a very simple AI.

The group designed \langname{} with the target users' ease of learning and ease of writing in mind, and the resulting language closely resembles a character sheet. 

The process of designing the language was not as well planned as it could have been, however the group achieved the goals stated in the problem statement.

The implementation of the compiler proved harder than expected with certain constraints within the language suddenly becoming a problem, especially with regards to scope and type checking, however the group both has a functioning compiler and an engine to run the battle, which means that the project in its entirety was a success.