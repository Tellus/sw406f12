The programming language \langname{} is specifically developed to define characters, abilities, attributes, resources, items, behaviour and events. Which are found within the role-playing games mentioned in \vref{baseclasses}. Events are triggered by various conditions during a battle, which the programmer can specify.
To make use of \langname{} a \emph{game engine} is used, which will simulate a turn based fight between two characters defined with the language. The engine abides by a rulebook, where the core rules of the RPG-system are defined. 

The behaviour of a user-defined character can be described as A.I.(Artificial Intelligence). The character will 'consider' it's next move, by assessing the risk of available moves against an enemy, based on a rate value. The risk and reward of possible moves can be defined with \langname{}.

\section{Language syntax}
This section focuses at describing the general syntax of the \langname{}. The description includes class inheriting, declaration of members and examples of various capabilities of \langname{}.

\subsection{Classes}
Creating an entity in \langname{} is like inheriting a class in an object-oriented language.
In \langname{}, the classes to inherit from are base classes; \emph{Character, Ability, Attribute, Resource, Item, Behaviour}. This helps make sure that the engine 'knows' how to deal with various user-defined units.
When inheriting in \langname{}, two keyword are used; \emph{make} and \emph{from}
\begin{lstlisting}
	make Human from Character
	{
		//Some members are put here
	}
\end{lstlisting}
In the example above the 'class' \emph{Human} inherits from the predefined 'base class' \emph{Character}.

\subsubsection*{Members in classes}
In \langname{}, the members are declared as collections, by using the symbols ' [ ' and ' ] ', separated with a comma ' , '.
A line is ended with the symbol ' ; '.
This can be enclosed in blocks, by using the symbols ' \{ ' and ' \} '.
\begin{lstlisting}
	make Human from Character
	{	
		Abilities:
			[ Pray, 
			  Heal,
			  Attack ];
	}
\end{lstlisting}
The example above is a \emph{Human} class, with its members, a collection of abilities.

\subsubsection*{References}
References in \langname{} are declared by using the keywords \emph{owner} and \emph{this} and the operator \emph{' . '} The keyword \emph{this} refers to the class, in which it is declared. The keyword \emph{owner} refers to the base class, which the class is derived from. 
\begin{lstlisting}
	make Human from Character
	{
		Health: 100;
	}
\end{lstlisting}
The example above is a \emph{HealthPoints} resource, which can be assigned to a character to keep track of their health. The keyword \emph{this} is used to 