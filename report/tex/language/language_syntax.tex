The programming language \langname{} is specifically developed to define characters, abilities, attributes, resources, items, behaviour and events. Which are found within the role-playing games mentioned in \vref{baseclasses}. Events are triggered by various conditions during a battle, which the programmer can specify.
To make use of \langname{} a \emph{game engine} is used, which will simulate a turn based fight between two characters defined with the language. The engine abides by a rulebook, where the core rules of the RPG-system are defined. 

The behaviour of a user-defined character can be described as A.I.(Artificial Intelligence). The character will 'consider' it's next move, by assessing the risk of available moves against an enemy, based on a rate value. The risk and reward of possible moves can be defined with \langname{}.

\section{Language syntax}
This section focuses at describing the general syntax of the \langname{}. The description includes a comparison of \langname{} and the `C++' language, especially the parallels between class-inheriting in the two programming languages. 

\subsection{Classes}
Defining a character in \langname{}, is like inheriting a class in C++.
Lets see an example of inheriting in C++:
\begin{lstlisting}
	class Animal
	{
		//Some members are put here
	}
	
	class Cat :: Animal
	{
		//Some members are put here
	}
\end{lstlisting}
In the example above the class \emph{Cat} inherits from the class \emph{Animal}. In \langname{}, the classes to inherit from are base classes; \emph{Character, Ability, Attribute, Resource, Item, Behaviour}. This helps make sure that the engine 'knows' how to deal with various user-defined units.
When inheriting in \langname{}, two keyword are used; \emph{make} and \emph{from}
\begin{lstlisting}
	make Human from Character
	{
		//Some members are put here
	}
\end{lstlisting}
In the example above the `class' \emph{Human} inherits from the predefined 'base class' \emph{Character}.

\subsubsection*{Members in classes}
Like in C++, in \langname{} it is possible to make members of a class by enclosing the members in a block, by using the symbols ``\{'' and ``\}''.
\begin{lstlisting}
	class Animal
	{
		string name;
	}
\end{lstlisting}
In the example above, the class \emph{Animal} has the member \emph{name} which is a string, and the line is ended by the symbol `;'.
In \langname{}, the members are declared as collections, by using the symbols ``['' and ``]'', and separate the members of the collections with a comma `,', and ending the line with the symbol `;'.
\begin{lstlisting}
	make Human from Character
	{	
		Abilities:
			[ Pray, 
			  Heal,
			  Attack ];
	}
\end{lstlisting}
The example above is a \emph{Human} class, with its members, a collections of abilities.

\subsubsection*{References}
In C++, it is possible to make references to the classes, with the keywords \emph{this} and \emph{parent} and the operator \emph{`-$>$'}.
\begin{lstlisting}
	class Animal
	{
		string name;
		this->name = "Fox";
	}
\end{lstlisting}
 References in \langname{} are declared by using the keywords \emph{owner} and \emph{this} and the operator \emph{`.'} The keyword \emph{this} refer to the class the keyword declared in and the keyword \emph{owner} refer to the super class, which is the top of the inheriting. 
\begin{lstlisting}
	make HealthPoints from Resource
	{
		this.health: 100;
	}
\end{lstlisting}
The example above is a \emph{HealthPointssets} resource, which can be assigned to a character, to keep track of its health.