\begin{comment}
\section{EBNF}
To explain the language syntax, an \emph{EBNF} notation is used.
EBNF is an abbreviation for \emph{Extended Backus-Naur form} and is a modified version of \emph{BNF} (Backus-Naur form).
EBNF is a notation technique to describe the syntax of languages and is often used for computer programming languages. EBNF is a set of derivation rules, meaning that a 'word' on the left side (called a non-terminal) is substituted with the words on the right side.\\
Example:\\
{\begin{lstlisting}[numbers=none]
full-name = name, last-name, $ | name, full-name

name> = first-name | middle-name
\end{lstlisting}}
Translation:
\begin{itemize}
	\item A full-name consists of a name, followed by a last-name and end-of-line, or a name, followed by full-name.
	\item A name consists of a first-name or a middle-name
\end{itemize}
\pagebreak
\end{comment}

\pagebreak
\section{\langname{} EBNF}
\begin{comment}
Two criteria were used to support the design of our language syntax, these are writability and readability. Writability is a measure for how convenient, for a programmer, a language is to write and readability is a measure for how easy a language is to understand and learn. These two have an extreme on each side of the scale, where a complex language can be extremely efficient due to type security, the number of operations and data structures, on the other side a simple one, made to be easy to use and learn often sacrifices some of the type security, operations and data structures to be more convenient for the user.
\end{comment}
%Here follows the \ac{ebnf} notation for \langname{}:
{\begin{lstlisting}[numbers=none]

program = "begin" , { declaration } , $ ;

declaration = "make", identifier, "from",  identifier , "{", { assignment } , "}" ;

assignment = identifier , ( ":" | "+:" | "-:" ) , value , ";" ;

value = expression | set | call;

expression = term , [ ( "+" | "-" ) , expression ] ;

term = factor , [ ( "*" | "/" ) , term ] ;

factor = single_value | ( "(" , expression , ")" ) ;

single_value = bool | number | string | class_member | value_keyword ;

class_member = identifier | reference_keyword , { "." , identifier } ;

set = "[" value , { "," , value } , "]" ;

reference_keyword = "owner" | "this" ;

value_keyword = "self" | "enemy" | "ally" | "team_ally" | "team_enemy" | "all" ;

call = identifier , "(" , value , { "," , value } , ")" ;

identifier = ( letter | "_" ), { letter | digit | "_" };

string = '"'{ all_symbols }'"' ;

bool =  "true" | "false";

number = {digit}, [ ".", {digit}] ;

digit = "0" | "1" | "2" | "3" | "4" | "5" | "6" | "7" | "8" | "9" ;

letter = ? A - Z and a - z ?

all_symbols = ? all symbols besides " ?

\end{lstlisting}}
\pagebreak
\begin{itemize}
\item program is the code to be compiled, from the beginning statement to the end of input.
A program may contain one or more declarations

\item declaration is the declaration of entities in \langname{}, after analysing the target group
and deciding upon readability over writability, the words "make" and "from" were chosen.
A declaration takes two identifiers and one or more assignments

\item assignment is the assignment of a value to a specific identifier, the symbol : was chosen
to resemble a character sheet.
An assignment takes an identifier, an operator and a value, followed by the symbol ;

\item value consists of either an expression, a set or a call

\item expression is a term, optionally followed by an operator and another expression

\item term is a factor, optionally followed by an operator and another term

\item factor is either a single\_value or an expression

\item single\_value can be a bool, number, string, class\_member or a value\_keyword

\item class\_member is an identifier or a reference\_keyword, optionally followed
by the symbol . and another identifier. The optional identifier is to specifically access members
of a given identifier

\item set is a list of values, contained within brackets "[" and "]"

\item reference\_keyword is either "owner" or "this", for writability these reference keywords are included
to make it easier for the user to refer to either a superclass (owner) or the active object (this)

\item value\_keyword is much like reference\_keyword but has to be given definitions of each keyword.
value\_keyword is can be "self", "enemy", "ally", "team\_ally", "team\_enemy" or "all".
These keywords are included to make it easier to define who are effected by specific actions

\item call takes an identifier, followed by one or more value enclosed in parantheses "(" and ")"

\item identifier starts with either a letter or an underscore "\_", followed by either letter, digit or
an underscore

\item string consists of all\_symbols enclosed in quotation marks '"'

\item bool is a boolean expression which can be either false or true

\item number is the only form for representing numbers in \langname{}, in a form of a decimal fraction
this form is both more detailed than an integer and eliminates the need for casting when calculating

\item digit represents one symbol from the decimal system, namely 0 - 9

\item letter represents one symbol from the american alphabet, both uppercase and lowercase

\item all\_symbols is every symbol in the ascii table, besides the quotation mark '"'

\end{itemize}