\subsection{Inclusions}
A feature of Primarch and seed definitions, inclusions is a notation to define a member of a Primarch without explicitly instantiating with an identifier. An example:
\begin{lstlisting}[language=fflang]
make Attack from Ability
{
	targets: [enemy];
	effects: [	PhysicalDamage(target,12)
			 ];
}

make Cure from Ability
{
	targets: [self];
	mana_cost   : 7
	effects: [Heal(target,28)];
}

make BattleMageBehaviour from Behaviour
{
	positive : [[owner.health, 70]];
	negative : [[enemy.health, 80]];
}

make Wizard from Character
{
	abilities : [Cure, Attack];
	behaviour : BattleMageBehaviour;
	events    : [ ];
                 
    strength: 8;
    agility: 8;
    intelligence: 21;
    
    health: 80;
    mana: 100;
}
\end{lstlisting}

In this example, Wizard is made from the Primarch Character, and is assigned new abilities, values of attributes and resources as well as a behaviour. Notice that unlike imperative languages, only the identifier is given - no types are necessary, since each entry of the type is considered unique: It is impossible to have two Strength attributes on the same character.