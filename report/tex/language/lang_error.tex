\section{Error reporting}

One of the most important things to be aware of when designing and implementing a compiler is how the programmer is informed of errors. Errors in programming are very common, errors are almost always user-related like grammatical errors, trying to use unsupported operations, referring to undeclared entities, endless loops, memory misuse etc.

\subsection{Error types}
Errors can be caught in different phases of compiling.
While scanning and/or parsing the compiler can report syntax errors like spelling mistakes, missing symbols, incorrect use of operators and missing package imports (references to code that the user needs to include to use).\\
A user might for example write "whihle" in his control statement, where "while" is the right expression. The compiler might then return "unrecognised expression "whihle" on line <line number>". With this information the user can then lookup the error on the given line number and correct the mistake.\\\\
While type-checking the compiler can report type errors like undeclared references, use of undeclared variables, double declarations, operation on invalid types etc.\\
A user might for example try to assign a value to a variable which has no type.
{\begin{lstlisting}[numbers=none]
/* Variable with no type */
x = 5;

/* Variable with a type */
int x = 5;
\end{lstlisting}}
The level of detail is of course up to the designers of the compiler, as the errors that are reported are in the hands of the designers, not the compiler itself.