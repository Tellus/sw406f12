\section{Language constructs}
The \langname{} language has resulted in a highly declarative style of programming, focusing on what is to be done, but not how. While several of the concepts featured in the language have well-known parallels in other programming languages (events/signaling, classes, inheritance, user-defined types), they have been modified to more specifically target the domain of role-playing games.

The very first few iterations of \langname{} presented a hybrid attempt between the implicitly understood effects of declarative programming with the flow control of imperative programming. As analysis of the target audience took hold, however, it became clear that the language should provide for simplistic input of objects \todo{classes?} and their relationships, without the hassle of considering line ordering or even algorithms. The focus should not be on having the programmers define the sometimes tedious rule sets and calculations of the games \todo{Reference WoD}, but rather let them take the natural role of a \ac{dm} and define not the world, but its contents. This has resulted in the unavoidably imperative elements of the language (in particular the rule set and base derivations of types) to be segregated into pre-written $C++$ code (the external ruleset) while the definition of characters and the setting of their attributes remains within \langname{}. This section \todo{change if we go higher/lower} describes the primary features of the \langname{} language as well as the underlying reasoning behind them.
