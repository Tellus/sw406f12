\section{Language constructs}
The \langname{} language is tailored for a highly declarative style of programming, focusing on what is to be done, but not how. Several of the concepts featured in the language have well-known
parallels in other programming languages e.g. events/signaling and user-defined types. The concepts have been chosen to give the best possible representation of the flow of battle in RPG's.

After analysing the target audience, it became clear that the language should provide for
simplistic input of objects and their relationships, without considering line
ordering or algorithms. The focus should not be on having the programmers define the rule-sets
and calculations of the games, but rather let them take the natural role of a \ac{gm}
and define not the world, but its contents. The rule set was originally to be described in the
language, but this presented too big a challenge to stay within scope of the project and the rule set was instead included in the engine, while the definition of characters and the setting of their attributes remains within \langname{}. This section describes the primary features of the
\langname{} language as well as the underlying reasoning behind them.
%\todo{classes?} \todo{change if we go higher/lower}