\subsection{Types and user-defined types}
\label{language:types}
The core concepts of any roleplaying game necessitates a very rigid data structure, and the basis for \langname{}'s reference rule set is no different.
\langname{} defines a number of generalised base types, derived from the concepts found through most of the \ac{rpg} systems analysed. They are, briefly, as follows:
\begin{description}
	\item[Character] defines any acting entity with an invididual turn sequence.
	\item[Resource] Represents values, like health or mana, that can have a minimum and maximum value.
	\item[Attribute] Represents any numeric value associated with the Character, this could be; Strength, Intelligence an so on.
	\item[Behaviour] sets the values used to calculate a decision tree for a Character.
	\item[Event] represents a reactive AI routine. Complements Behaviour.
	\item[Ability] is any action known to a Character.
	\item[Effect] types are associated with Ability types and define what happens when an Ability is used.
	\item[Item] describes consumable or equippable items in the game, that may rise the current value of an Resource or Attribute, such as a sword or a potion.
\end{description} 

All of the types except Event and Behaviour were derived from the combined concepts of the systems analysed. All the systems contain actors, in this case a Characters, with core characteristics (Attributes and Resources). Depending on their archetype (or class or equivalent \ac{rpg} concept), the Characters have differing sets of abilities, which is the same as powers in \ac{dnd} 4th edition and disciplines in \ac{vtm}.
Abilities will always have the capacity to affect the game in some way (Effects), although they will not always be successful or fully potent (see \vref{language:ruleset}). An Ability will almost always have an associated cost. If not time (a turn), some resource must be expended in order to execute the Ability. Finally, Items are an almost universal feature, either in the form of useable objects (which, in turn, expend some form of resource of the Character or the Item) or as an equippable (a sword, a helmet, a ring or similar).

Event and Behaviour are defined as a means for the programmer to describe how Characters should act within the scenario. While it is very rare for a pen and paper rulebook to describe character behaviour in full play by play, they do often describe character traits that can be used to derive behaviour. This description is necessary for a later AI simulation to make meaningful decisions (see the section on AI theory \todo{Write that section...}). Conversely, all video \ac{rpgs} today demonstrate some form of AI in enemy characters, however simple.

In order to make use of the types, the programmer must define a named version of that type. Although it is difficult to define this mechanism as either instantiation (object of class) or inheritance (subclass of class), it mostly resembles the prior. The programmer must define a type (say, a new Ability - Fireball) strictly following the data structure of its source (Ability). \langname{} does not support extending the structure of any type. For example, the Fireball Ability cannot add a member named "colour". It is, however, possible to further derive from a named type (creating a BigFireball from Fireball, for example) wherein select member values have been modified. Extending the structure of a type is not impossible, however. The seven archetypes are defined in the core of \langname{}, but the rule set is able to extend the structure as may be necessary for an \ac{rpg} system. Another trait evident from this type derivation (and member assignment) is that type names and variables go into one. Once the Ability derivatives Fireball and BigFireball have been declared, they can be inserted into a Character's Ability collection via assignment.

This combination of concepts further simplifies the work involved in creating new content for a well-defined rule set. Any \langname{} programmer for a system merely needs to know the rules and which conventions were used to embed them into \langname{} from the rule set, (re)defining as necessary.
Again the reasoning falls back to the targeted end users and the focus of creating simple constructs with enough flexibility to define desired scenarios. Named variables are unnecessary as no more than one instance of any specific derivative was never found to be necessary, multiple consumable Items was a possible consideration, but can be handled via a Resource or Attribute that defines number of available uses.
%The resulting localised Singleton pattern or nameless object (while Fireball is only defined one place, each occurrence on a new Character constitutes an individual instance) much closer resembles the typical method of filling out data in a character sheet - it is very rare indeed to see a system where the player is required to come up with their own name for an Ability, not least because it makes the \ac{dm}'s task much more complex, and needlessly so.
