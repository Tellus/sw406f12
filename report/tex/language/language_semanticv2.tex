\section{\langname{} Semantics}
%I don't like this anymore :/ ~Hornbjerg

%{\Large[The real stuff]}
The semantics of \langname{} are based on the environment-store model, described in section \ref{subsec:env-sto}.
The semantics are split into two parts; declarations in \langname{}, and the inner workings of the engine used to simulate a battle between two characters.
\subsection*{Declaration Semantics}
For the purpose of semantics, a simplified syntax is assumed, where every declaration follows a specific line ordering. This means that the line agnosticism mentioned before is ignored in this section.
In the semantics of \langname{}, \ac{rgr} is a set of keywords used to refer to characters. These have been limited in contrast to the language's original syntax.\\ \textbf{RGR} is defined as follows;  
%
$$\mathbf{RGR} = \{\texttt{owner, target, enemy\}}$$
%
Relative Global References are used in the declaration of abilities and behaviour. An ability is declared by a name and contains the targets to be affected by the ability, effects carried out by the ability and three values. 
 The first value in the Ability environment is a list of possible targets, followed by the \ac{hp} and Mana cost and finally a list of effects with their respective targets and magnitude.\\
To keep track of the bindings of an ability, an \textit{ability environment} is used. The set of ability environments is called \textbf{Ability}. 
Given an ability name \textit{AName} $\in$ \textbf{AName}, which is the set of legal ability names, an ability environment must hold information about, targets, costs and effects. Therefore the definition of the set of ability environments \textbf{Ability} is:
%
$$\mathbf{Ability: \; AName \rightharpoonup \mathcal{P}(RGR) \times Number \times Number \times \mathcal{P}(Effect \times RGR \times Number)}$$

As mentioned, \ac{rgrs} are used for behaviour as well. A behaviour is declared by a name and it contains a list of 3-tuples ({\textit{RGR} $\in$ \textbf{RGR}}, \textit{VName} $\in$ \textbf{VName}, \textit{Num} $\in$ \textbf{Number}), where \textit{\ac{rgr}} is a reference to the involved character, \textit{VName} is the name of the involved attribute and \textit{Num} is a value representing the priority of the attribute. To keep track of the bindings of a behaviour, a \textit{behaviour environment} is used. The set of behaviour environments is called \textbf{Behaviour}.
A behaviour environment must, given a legal behaviour name \textit{BName}, hold information of the list of 3-tuples. Therefore the definition of the set of behaviour environments is:

$$\mathbf{Behaviour: \; BName \rightharpoonup \mathcal{P}(RGR \times VName \times Number)}$$
%\\\\
For declaring forced actions for a character, \langname{} makes it possible for a user to declare events, which contain a condition(health below 10\%), an ability to execute when the conditions are met and a target for this ability.
An \textit{event environment} is used to keep track of the binding for events. The set of event environments is called \textbf{Event}.
Given a legal event name, \textit{EName} the event environment must hold information about the condition, target and ability. Therefore the definition of the set of event environments is:

$$\mathbf{Event : \; EName \rightharpoonup Condition \times RGR \times AName}$$

%\\\\
When a character is declared, they are declared with some resources, \ac{hp} and Mana. The resources have a maximum and current value that can be set, while the minimum is set to 0. A resource therefore has bindings for the two values. To keep track of these bindings a \textit{resource environment} is used.\\
The set of resource environments is called \textbf{Resource}, and is defined as follows:

$$\mathbf{Resource : \; RName \rightharpoonup Number \times Number}$$

There are no written transition rules or systems for \textbf{Resource}, because resource is not declared standalone. Resources are declared as soon as a character is declared.
It is also required to keep track of the bindings of the character itself. To keep track of the bindings a \textit{character environment} is used. The set of character environments is called \textbf{Character}.
Given a character name \textit{CName} or an \textit{RGR}, a character environment must hold information about the behaviour, abilities, attributes, events and resources bound to the character. Therefore the definition of the set of character environments \textbf{Character} is:
%
$$\mathbf{Character} : \; \mathbf{CName} \cup \mathbf{RGR} \rightharpoonup \mathbf{BName} \times \mathbf{\mathcal{P}(AName)} \times \mathbf{Env} \times \mathcal{P}(\mathbf{EName}) \times \mathbf{Resource}$$

\textbf{Env} is an environment used for keeping track of bindings between attributes and resources, and locations. \textbf{Env} is defined
\[ 
\mathbf{Env} = \mathbf{VName} \rightharpoonup \mathbf{Loc}
\]

All declarations, except character declaration, in \langname{} have the same transition system. 
%All declaration in \langname{} have the same 
%The declarations in \langname{} have the same transition system.
The transition system is as follows:\\\\
\begin{tabular}{l l}
$\Gamma_{Dec} = $ & $\mathbf{Character \times Ability \times Behaviour \times Event \times Sto \times Stm}$ \\
$\Rightarrow_{Dec}$ & \\
$T_{Dec} = $ & $\mathbf{Character \times Ability \times Behaviour \times Event \times Sto}$ \\
\end{tabular}
\\\\
The \textbf{Stm} varies after what type is being declared. The statement (\textbf{Stm}) is described for each type of declaration before their rules are presented. Note that environments are updated according to the declaration type.\pagebreak

The \textit{Stm} for an Ability declaration is $$\mathbf{AName \times \mathcal{P}(RGR) \times Number \times Number \times \mathcal{P}(Effect \times RGR \times Number)}$$

\begin{table}[!h]
\begin{tabular}{l l}
\\ \hline \\
\small{\textsc{[Ability-Dec]}}& \\
\footnotesize{$\langle Character, Ability, Behaviour, Event; \; AName,  RGR, ManaCost, HealthCost,$}
 & \footnotesize{$Effect_1, RGR_1, Magnitude_1$} \\
 &  \footnotesize{$Effect_2, RGR_2, Magnitude_2$}\\
 &  \footnotesize{$Effect_3, RGR_3, Magnitude_3$} \\
 &  \footnotesize{$\vdots$} \\
 &  \footnotesize{$Effect_n, RGR_n, Magnitude_n \rangle \; \Rightarrow$} \\
\end{tabular}
\begin{tabular}{l p{0.92\textwidth}}
 \footnotesize{$\langle Character, Ability', Behaviour, Event \rangle$} \\
 \footnotesize{Where $Effects = $} \footnotesize{${\bigcup^{n}_{i = 0}} (Effect_i, RGR_i, Number_i)]$} \\
 \footnotesize{And} $Ability' =$ \footnotesize{$Ability[AName \mapsto (RGR, ManaCost, HealthCost, Effects)]$} \\
\\ \hline
\end{tabular}
\caption{Transition rule for \texttt{make} $AName$ \texttt{from Ability}}
\label{tbl:decA}
\end{table}

The \textit{Stm} for Behaviour is as follows: $$\mathbf{BName \times \mathcal{P}(RGR \times VName \times Number)}$$

\begin{table}[!h]
\begin{tabular}{l l l}
\\ \hline \\
\small{\textsc{[Behaviour-Dec]}} & \\
 & \footnotesize{$\langle Character, Ability, Behaviour, Event; BName,$} & \footnotesize{$RGR_1, VName_1, Factor_1$} \\
 & & \footnotesize{$RGR_2, VName_2, Factor_2$}\\
 & & \footnotesize{$RGR_3, VName_3, Factor_3$} \\
 & & \footnotesize{$\vdots$} \\
 & & \footnotesize{$RGR_n, VName_n, Factor_n  \rangle \; \Rightarrow$} \\
\end{tabular}
\begin{tabular}{l p{0.85 \textwidth}}
 & \footnotesize{$\langle Character, Ability, Behaviour', Event \rangle$} \\ %& \color{white} \footnotesize{$RGR_1, VName_1, Number_1 \Rightarrow$} \\
\footnotesize{Where $Behaviour' = $} & \footnotesize{$Behaviour[BName \mapsto {\bigcup^{n}_{i = 0}} (RGR_i, VName_i, Factor_i)]$} \\
\\  \hline 
\end{tabular}
\caption{Transition rule for \texttt{make} $BName$ \texttt{from Behaviour}}
\label{tbl:decB}
\end{table}

The \textit{Stm} for Events is: $$\mathbf{EName \times Condition \times RGR \times AName}$$
%Event
\begin{table}[!h]
\begin{tabular}{l l}
 \\ \hline \\
 \small{\textsc{[Event-Dec]}} & \\
 & \footnotesize{$\langle Character ,Ability, Behaviour, Event; \; EName, Condition, RGR, AName \rangle \; \Rightarrow$} \\
 & \footnotesize{$\langle Character, Ability, Behaviour, Event' \rangle$} \\
\footnotesize{Where $Event' =$} & \footnotesize{$Event[EName \mapsto (Condition, RGR, AName)] $} \\
 \\ \hline
\end{tabular}
\caption{Transition rule for \texttt{make} $EName$ \texttt{from Event}}
\label{tbl:decE}
\end{table}

%Resource
\pagebreak
The transition system for declaration of character is: \\
\\
\begin{tabular}{l l}
$\Gamma_{DecC} = $ & $\mathbf{Character \times Env \times Resource \times Ability \times Behaviour \times Event \times}$ \\
 & $\mathbf{CName \times BName \times \mathbf{\mathcal{P}(AName)} \times \mathcal{P}(\mathbf{EName}) \times Sto \cup \texttt{\textbf{\{next\}}}}$\\
$\Rightarrow_{DecC}$ & \\
$T_{DecC} = $ & $\mathbf{Character \times Env \times Sto}$ \\
\end{tabular}
\\\\

$\Rightarrow_{DecC}$ is defined by the rules in table \ref{tbl:decC}\\\\
\pagebreak
\begin{landscape}
\begin{table}[!h]
\begin{tabular}{l c}
\\ \hline \\
\small{\textsc{[Char-Dec]}} \\
 & \footnotesize{$env', sto''' \vdash A_D \Rightarrow sto'' \;\;\;\; env', sto'' \vdash R_D \Rightarrow sto'$}\\
  & $\frac{Character \vdash ANames, BName, ENames \Rightarrow Character'}{\langle Character, Abitity, Behaviour, Event, env, sto, \texttt{next} \; ; \; CName, A_D, R_D, ANames, BName, ENames \rangle \Rightarrow Character', env', sto'}$\\
\end{tabular}
\\
\begin{tabular}{l p{1.14\textwidth}}
\\
 \footnotesize{Where} $env'=$ &  \footnotesize{$env[health \mapsto l, \texttt{next} \mapsto \texttt{new } l,$} \\
 & \footnotesize{$mana \mapsto \texttt{new } l, \texttt{next } \mapsto \texttt{new (new } l),$} \\
 & \footnotesize{etc.$]$} \\
 \footnotesize{And} $sto''' =$ & \footnotesize{$sto[env'(health) \mapsto std,$}\\
 & \footnotesize{$env'(mana) \mapsto std,$}\\
 & \footnotesize{etc.$]$} \\
 \footnotesize{And} $Character'=$ & \footnotesize{$Character[CName \mapsto ANames, BName, ENames, env', Resources]$}
 \\ \hline
\end{tabular}
\caption{Transition rules for \texttt{make} $CName$ \texttt{from Character}. etc. stands for all Attributes for the character.}
\label{tbl:decC}

\end{table}	
\end{landscape}
The declaration of attributes and effects are not described above. As it can be seen i table \ref{tbl:decC} an environment $env$ is used, which is the environment used to keep track of bindings of attributes.

It is not possible for a user to declare custom effects. The effects are hard-coded in the engine and therefore the declaration of effects is not described in the semantics of \langname{}.

After all declarations have been performed, the battle-engine is started with the two declared characters set as opponents in a battle scenario.

\subsection*{\langname{} battle-engine}

The battle-engine for \langname{} runs in turns, each turn is executed if a certain win-condition is not met. 
The win-condition is (like effects) hard-coded in the engine, and it is therefore not possible, for a user, to specify in the \langname{} language. In the current implementation, the win-condition is met if the enemy of the current character has zero or less \ac{hp}.

At the start of each turn, the win-condition is checked. If it is met, the battle ends, else a turn is executed in the following way;

\begin{description}
\item[1] The character who is to be taking action the current turn is chosen.
\item[2] It is checked whether an event is triggered, if not the turn continues to point \textbf{3}, else the forced action is executed and point \textbf{3} and \textbf{4} are skipped.
\item[3] Based on the declared behaviour, the best action to execute is calculated and chosen, by simulating every possible action.
\item[4] The chosen action is executed.
\item[5] A new turn starts and the win-condition is checked.
\item[5] If the win-condition is not met the turn continues from point \textbf{1}.
\end{description}


The transition system for execution of a turn is

\begin{tabular}{l l}
$\Gamma_{TURN} = $ & $\mathbf{Character \times Behaviour \times Ability \times Event \times Resource}$\\
 & $\mathbf{\times CName \times CName \times Sto}$ \\
$\Rightarrow_{TURN}$ & \\
$T_{TURN} = $ & $\mathbf{(CName \times CName \times Sto) \cup \texttt{\{Abort\}}}$ \\
\end{tabular}
\\\\
$\Rightarrow_{TURN}$ is defined by the rules in table \ref{tbl:Turn}.\\\\

\begin{table}[!h]
\begin{tabular}{l l}
\\ \hline \\
\small{\textsc{[Turn$_1$]}} & \\
 & \footnotesize{$\langle Character, Behaviour, Ability, Event, Resource, Active, Next, sto \rangle \Rightarrow_{TURN} \langle Next, Active, sto' \rangle$} \\
% \\
\footnotesize{Where} & \footnotesize{$Character(Active) = (BName, AName, env, EName, Resource)$} \\
\footnotesize{And} & \footnotesize{$env(Health) > 0$ (win-condition not met)} \\
 & \footnotesize{$\forall e \in EName, (Condition, Ability) = Event(e), Condition \rightarrow_b f \! \! f$} \\
 & \footnotesize{$Be = Behaviour(BName)$} \\
 & \footnotesize{$sto' = ChooseAction(Character, Ability, Active, Next, Be, sto)$} \\
 \small{\textsc{[TURN$_2$]}} & \\
 & \footnotesize{$\frac{ \langle Character, Ability, Active, Event(e)RGR, Event(e)AName, sto \rangle \Rightarrow_{A\_exe} = sto' }{ \langle Character, Behaviour, Ability, Event, Resource, Active, Next, sto \rangle \Rightarrow_{TURN} \langle Next, Active, sto' \rangle}$} \\
% \\
\footnotesize{Where} & \footnotesize{$Character(Active) = (BName, AName, env, EName, Resource)$} \\
\footnotesize{And} & \footnotesize{$Env(Health) > 0$ (win-condition not met)} \\
  & \footnotesize{$\exists e \in EName, (Condition, Ability) = Event(e), Condition \rightarrow_b t \! \! t$} \\
% & \footnotesize{$Character(Active), Character(Next), Ability \vdash \langle Event(e)AName, sto \rangle \Rightarrow_{A\_exe} = sto'$} \\
 \small{\textsc{[TURN$_3$]}} & \\
  & \footnotesize{$ \langle Character, Behaviour, Ability, Event, Resource, Active, Next, sto \rangle \Rightarrow_{TURN} \langle \texttt{Abort} \rangle$} \\
%\\
\footnotesize{Where} & \footnotesize{$Character(Active) = (BName, AName, env, EName, Resource)$} \\
\footnotesize{And} & \footnotesize{$Env(Health) \leq 0$ (win-condition met)} \\
\\ \hline
\end{tabular}
\caption{Transition rules for $\Rightarrow_{TURN}$}
\label{tbl:Turn}
\end{table}
During a turn an ability is executed. The execution of an ability changes the values of the one (or both) of the involved characters .
All abilities have Mana and \ac{hp} cost, default set to zero. When an ability is executed, the cost of that given ability is subtracted from the Mana and \ac{hp} pool of the executing character. \\

The transition system for execution of an ability is

\begin{tabular}{l l}
$\Gamma_{A\_exe} = $ & $\mathbf{Character \times Ability \times CName \times RGR \times AName \times Sto}$ \\
$\Rightarrow_{A\_exe}$ & \\
$T_{A\_exe} = $ & $\mathbf{Sto}$ \\
\end{tabular}
\\\\
$\Rightarrow_{A\_exe}$ is defined by the rules in table \ref{tbl:abi-exe}.
\begin{table}[!h]
\begin{tabular}{l l}
\\ \hline \\
\small{\textsc{[Ability-Exe]}} & \\
& \footnotesize{$ \langle Character, Ability, Active, RGR, AName, sto \rangle \Rightarrow_{A\_exe} \langle sto' \rangle$} \\
\footnotesize{Where $sto''=$} & \footnotesize{$sto[Character(Active)Health -= Ability(AName)HealthCost,$}\\
 & \footnotesize{$Character(Active)Mana -= Ability(AName)ManaCost]$}\\
\footnotesize{And $sto'=$} & \footnotesize{$\forall ef \in Ability(AName)Effects:$}\\
 & \footnotesize{$ef(Character(Active), Character(ef.RGR), ef.Magnitude), sto''$}\\
\\ \hline
\end{tabular}
\caption{Transition rules for $\Rightarrow_{A\_exe}$}
\label{tbl:abi-exe}
\end{table}

Every ability causes an effect, these effects can not be defined in \langname{}. Instead the possible effects are predefined in the battle-engine. 
The possible effects are \textit{physical damage} and \textit{heal}. The notation $Effect(Character_1, Character_2, Number)$ denotes a function call, applying the effect to the target($Character_2$), using the attributes of both characters and the magnitude ($Number$) for calculations, and returns the updated $sto$.\\
\begin{comment}
The transition rules for execution of an effect are of the form
{\small $$EName \langle sto, l \rangle \Rightarrow_{eff} sto[l \mapsto x \pm i]$$}

The transition system for execution of an effect is 

\begin{tabular}{l l}
$\Gamma_{eff} = $ & $ \mathbf{Sto \times RGR, \times RGR \times Number \times Character}$ \\
$\Rightarrow_{eff}$ & \\
$T_{eff} = $ & $\mathbf{Sto}$ \\
\end{tabular}
\\\\
$\Rightarrow_{eff}$ are defined by the rules in table \ref{tbl:physical}.\\

\begin{table}[!h]
\begin{tabular}{l p{0.8\textwidth}}
\\ \hline \\
\small{[\textsc{Physical-Dmg}]} & \\
 & \footnotesize{$Character, sto \vdash Physical \langle  source, target, i \rangle \Rightarrow_{eff} sto'$} \\ %[l \mapsto x - i]$} \\
 \footnotesize{Where} & \footnotesize{$sto' = sto[target(Health) -\!\!= (source(attack\_power)- target(defense)) \times i]$} \\
 \\
\small{[\textsc{Heal}]} & \\
 & \footnotesize{$Character, sto \vdash Heal \langle source, target, i \rangle \Rightarrow_{eff} sto'$} \\
  \footnotesize{Where} & \footnotesize{$sto' = sto[target(Health) +\!\!=  i]$}
 \\ \hline
\end{tabular}
\caption{Transition rule for $\Rightarrow_{eff}$}
\label{tbl:physical}
\end{table}
\end{comment}

If no event is triggered during a turn, the action to be executed has to be calculated based on the behaviour declared in \langname{}.
The calculations are done by an algorithm that would turn into a long description if it was to be described by a semantic transition rule. It is instead described by pseudo-code.
The algorithm calculates a value called \texttt{piggy}-value. The size of the \texttt{piggy}-value determines whether the action is good or not, the greater the value, the better the action.
The size of the \texttt{piggy}-value is considered, based on the ratios declared in the behaviour of the character.\\
The pseudo-code is as follows \\\\
%Behaviour description to come
\textsc{Choose-Action}\\
%\texttt{
\begin{algorithm}[H]
\KwData{Character, Ability, Active, Next, Be, Sto}
\KwResult{Updated Sto}

actions := Character(Active)ANames\;

max\_piggy := - $\infty$\;

\ForEach {a $\in$ actions }{
	\If{(Ability($a$).healthcost <= Character(Active).health \textbf{and} Ability($a$).manacost <= Character(Active).mana)}
	{
		\textbf{continue};
	}

    temp\_sto  := copy of Sto;
    
    temp\_sto := Ability-Exe(Character, Ability, Active, Ability($a$.RGR), $a$, temp\_sto);
    
    new\_piggy := 0;
    
    \ForEach{RGR, VName, Factor $\in$ Be }
    {
    	new\_piggy +=: Character(RGR).VName * Factor; Considering temp\_sto
    }
    
    \If{(new\_piggy > max\_piggy)}{
        max\_piggy := new\_piggy\;
        best\_sto := temp\_sto;
        }
    }
return best\_sto;
\end{algorithm}

The pseudo-code describes how every possible action is simulated, and based on the \texttt{piggy}-value, the best action is chosen and returned.


%\todo{describe behaviour}
%Test text..
\begin{comment}
Missing:
Behaviour
%Event/forced behaviour
%Win-condition
\end{comment}


%\pagebreak
%
%\todo{Not sure if this have to be in}
%%%%%%%%%%%%%
% Vi har nogle regler der skal beskrives med br�dtekst.
% Kan ikke lige huske hvilke der mangler lige nu.. Det h�ber jeg at Danny kan hj�lpe med at huske........
%%%%%%%%%%%%%