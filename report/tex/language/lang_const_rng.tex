\subsection{Ranges and random values}
The language has built-in support of random number generation, going so far as to allow variables with unknown actual values until they are used in an expression. These types were conceived to cater to the widely used concepts of dice throws and resources. Resources are quite often described with minimum and maximum values. Dungeons and Dragons, for example, define a character's minimum and maximum health points as a factor of their chosen profession (class) and Constitution attribute (a measure of physical endurance). Likewise, damage rolls (the combination of dice and modifiers used to calculate damage dealt from an attack) are formulated in the general form $xdyy+z$ where $x$ is the number of $yy$-sided die to roll, plus a constant $z$, determined by modifiers relevant to the roll (strength for melee, intelligence for magic, for example). These calculations and types of data are somewhat trivial to implement in imperative languages, but the sheer frequency of their use invites a vastly simplified syntax for the expression of the concept. 	Consider the following example: a damage roll defined as $3d8+2$ in C:
