\subsection{\langname{} Rule Set}
\label{language:ruleset}
Many systems have complex conditions or calculations for some of their aspects.
Analysing the target group's most popular systems, several rules were exposed that cannot be expressed simply in a generalised declarative manner.
For example, characters with a health pool reduced to a non-positive number are no longer able to act, but are not entirely out of the battle.
Given medical attention, the character can return to battle.
Another example is the health system of the revised Storyteller system from the \ac{wod} series of games. Three distinct types of damage cover the same seven points of health. Damage must be calculated on specific areas of these points, and must be removed via separate means.
Simpler examples of rules are damage calculations of most video \ac{rpgs},
where certain attributes will lower the damage sustained from specific categories of damage. In Final Fantasy I, which is the rule set used in this report's reference implementation, physical attack damage is determined from two formulae depending on the active character's class. For examples see the section \vref{language:implset}.

\subsection{Implemented set}
\label{language:implset}
\begin{center}
\begin{tabular}{|l l l|}
\hline
\multicolumn{3}{|c|}{\textbf{Attributes}}\\
\hline
Strength: & Manually Set & (used for Damage calculations)\\
\hline
Agility: & Manually Set	 & (used for Defense calculations)\\
\hline
Intelligence: & Manually Set & (used for MP and SpellDmg calculations)\\
\hline
Stamina: & Manually Set & (used for HP and Defense calculations)\\
\hline
\end{tabular}\\
\emph{All above listed attributes are manually set by the programmer.\\ These initial values are fundamental for a character in the implemented world.}
\end{center}

\begin{center}
\begin{tabular}{|l l|}
\hline
\multicolumn{2}{|c|}{\textbf{2nd Attributes (calculated)}}\\
\hline
Defense: & ((Stamina + Agility) / 3)\\
\hline
Magic Defense: & ((Stamina + Intelligence) / 3)\\
\hline	
\end{tabular}\\
\emph{These second attributes are calculated with the given values of the initial attributes.\\ They serve as defense for various damage types, by subtracting their assigned values from the damage taken.}
\end{center}

\begin{center}
\begin{tabular}{|l l|}
\hline
\multicolumn{2}{|c|}{\textbf{Resources}}\\
\hline
Health Points: & (Stamina * 20)\\
\hline
Mana Points: & (Intelligence * 15)\\
\hline
\end{tabular}\\

\end{center}

\begin{center}
\begin{tabular}{|l l|}
\hline
\multicolumn{2}{|c|}{\textbf{Abilities}}\\
\hline
Attack Power: & (WeaponDmg + ((Strength / 2) + 1))\\
\hline
Magic Power: & (Equipment bonus + ((Intelligence / 2) + 1))\\
\hline
\end{tabular}\\
\emph{These are values, calculated for use in various abilites, both harmful and helpful.}
\end{center}


\begin{comment}
Many systems have complex conditions or calculations for some of their aspects. Analysing the target group's most popular systems, several rules were exposed that cannot be expressed simply in a generalised declarative manner. For example, characters with a health pool reduced to a non-positive number are no longer able to act, but are not entirely out of the battle. Given medical attention, the character can return to battle.
Another example is the health system of the revised Storyteller system from the \ac{wod} series of games. Three distinct types of damage cover the same seven points of health. Damage must be calculated on specific areas of these points, and must be removed via separate means.
Simpler examples of rules are damage calculations of most video \ac{rpgs}, where certain attributes will lower the damage sustained from specific categories of damage. In Final Fantasy I, which is the rule set used in this report's reference implementation, physical attack damage is determined from one of two formulae depending on the active character's class. %($WeaponAttackPower + \frac{Strength}{2}$ for regular characters and either $WeaponAttackPower + \frac{Strength}{2} + 1$ or $Level \times 2$ for monks).
In order to accommodate these requirements, it would be necessary to either introduce complex conditionality or imperative state manipulation in \langname{}, or externalise these specific parts where they could be described more fully.
\end{comment}