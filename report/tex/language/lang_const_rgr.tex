\subsection{Relative Global References}
\label{relative}
Relative Global References are a set of keywords that correlate to a dynamic set of global references. These are \texttt{owner}, \texttt{enemy}, \texttt{target} to name a few, and will always reference the respective entity from the perspective of the local scope, \texttt{enemy}, for example, will always reference \textbf{another} Character in a battle when applied inside a Character subtype or its members, while \texttt{owner} always references the topmost Character in a membership hierarchy. Consider the following code:

\begin{lstlisting}[language=fflang]
make Selfpreserving from Behaviour
{
	positive: [owner.health, 100];
	negative: [enemy.health, 20];
}

make Wizard from Character
{
	//Some code omitted for clarity
	behaviour: Selfpreserving;
}

make Dragon from Character
{
	name: "Dagon";
}
\end{lstlisting}

\texttt{owner} when referenced in Selfpreserving, will point to whichever Character is considering an action. Since the Wizard has the behaviour Selfpreserving, \texttt{owner} points to Wizard and \texttt{enemy} points to Dragon. All of the RGR's thus automatically represent a relationship model, allowing the programmer to focus purely on the subtype at hand rather than the larger class architecture and the implications of n-element aggregation chains.
