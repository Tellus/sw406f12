The programming language `RPG-script' is specifically developed for defining characters and abilities, attributes, resources, items and behaviour belonging to a character and the user of the language can define events, which are triggered by some conditions and uses some ability and finally the user can define that an ability triggers some rulebook defined effect. 
In addition to the programming language is a game engine, which will simulate a turn based fight between the characters defined in the language. 
The behaviour of a user defined character is A.I.(Artificial Intelligence) based, which means it is based at a value called \emph{piggy rate}, the character will consider its possible moves, and by an user defined priority, decide the move that will give the character the highest piggy rate at the end of the turn.

\section{Language syntax}
This section focuses at describing the general syntax of the `RPG-script' languages. The description will include a comparison of the `RPG-script' language and the `C++' language, in specially the parallels between inheriting of classes in the two programming languages. 


To define a character in the language, the user makes some primarchs, which is superclass, which not inherit from any other classes \todo{Primarchs are a type of primitive (basest of base classes), the real implementation of which is based on the external ruleset file}. The primarchs will be used for defining characters, abilities, attributes, resources, items, behaviours and effects.
Each primarch have members which may be references to an other primarch. The members of the primarch is enclosed in a block by using the symbols ``\{'' and ``\}''. To define a primarch the keyword \emph{core} is used \todo{Outphased. All "core" Primarchs are defined externally, and RPG-script exclusively defines the subtypes/inheritance}.
Let's consider the following code example 
\begin{lstlisting}
	core Character //Make a primarch called Character
	{
	}
\end{lstlisting}
The code example above, makes a primarch with the name `Character'. 
To make member of a primarch, or class, a collections is used, which is declared by using the symbols ``['' and ``]'', and separate the members of the collections with a comma `,'.
\begin{lstlisting}
	core Character
	{
		Attributes:
			[ FireBall, //Make a collections of attributes as a member of the primarch
				Heal,
				Attack ]; 
	}
\end{lstlisting}
\emph{Fixed version here.}
\begin{lstlisting}
	make Wizard from Character
	{	
		// Attributes: // attributes, together with resources, are now defined exclusively in the external ruleset.
		Abilities:
			[ FireBall, //Make a collections of attributes as a member of the primarch
				Heal,
				Attack ]; 
	}
\end{lstlisting}
Because whitespaces has no function in the `RPG-script' language there is no deference between writing the code above, or writing:
\begin{lstlisting}
	core Character
	{
		Attributes:[ FireBall, Heal, Attack ];
	}
\end{lstlisting}
\begin{lstlisting}
	make Wizard from Character
	{
		Abilities:[ FireBall, Heal, Attack ];
	}
\end{lstlisting}
The language allow the user to inherit from a primarch or an other class, e.g. if the user want to make a `FireBall' ability, it can be made as a subclass of the primarch `Ability'. To inherit from a primarch, or an other class which not is a primarch, two keywords are used; \emph{make} and \emph{from} \todo{ALL inheritance is done this way, and only Primarchs (and their subtypes) can be inherited :)}.
Let's consider the following code example
\begin{lstlisting}
	make Strength from Attribute //Makes a subclass of Attributes which is named Strength
	{
		//Set relevant modifiers et al for this attribute.
	}
\end{lstlisting}

