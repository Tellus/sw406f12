\section{Artificial intelligence}
\ac{ai} is the term used for machine intelligence. \ac{ai} exists in many forms, amongst others; computer games, robotics and speech recognition.\\
Whether computers and machinery can be referred to as ''intelligent'' is debated due to the various definitions of intelligence and how they are compared to human intelligence.\\
The study of \ac{ai} is often related to intelligent agents, who are are described as systems that can react to their environment. Which means that an agent is able to calculate the most appropriate of actions available from analysing their environment and the changes that occur in it.\\
Some form of agents can be found in many video games, for example, strategy-, shooting- and adventure games.\cite{artint}

\subsection{Video game \ac{ai}}
The term \ac{ai} is often used within video games and generally refers to the programmed behaviour of non-player characters in video games, most often made to resemble human players. One of the ways they resemble humans is in the way the are able to make ''mistakes'', these so-called mistakes are intentional and in many cases required to provide fairness.\\
Imagine playing a game of hide and seek against a computer, the computer runs the world being played in, is fully aware of the environment and the changes in it, otherwise the computer could not represent it. This means that the computer is aware of your position in the game at all times, in other words: The computer will always find you. To make this game of hide and seek more interesting and fun for a person the programmer could, for example, make ''blind spots'' in the computer's searching algorithm.\cite{videoint}

\subsection{Action selection}
Action selection or decision making is a particular form of \ac{ai} that often is used in video games. Actions are selected by running a sort of business-model control structure, meaning that the non-player character which gets assigned the specific behaviour is presented for a set of actions to execute. Which action to execute is up to the control structure that lies within, like an ''if-else' structure it checks if certain conditions have been met and then responds with the appropriate action.\cite{actionselect}