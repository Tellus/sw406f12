In this chapter, the different language paradigms, along with the available tools and methods to define and design a language, are analysed. Since this language is targeted at role-playing games, an analysis of various role-playing games and systems is included.

\begin{comment}
\section{Analysis of Role-playing systems}

As mention the basic idea of a role-playing game is interactive story-telling. The game consists of several players and a \ac{gm}. The \ac{gm} is the mastermind behind the story being acted out, called the scenario. They are in charge of writing the plot of the scenario, and guiding the players through it, all the while allowing them to shape the individual encounters and scenes, but maintaining the greater story-arc. The players receive points for their characters surviving battles, using skills or sometimes as a reward to the player for excellent roleplaying of the character. Often dice are used to introduce randomness in determining the outcome of actions. The use of dice is incorporated into popular role-playing systems like Dungeons \& Dragons (DnD) and World of Darkness (WoD), both very popular pen and paper roleplaying systems.
\end{comment}

\section{Pen and Paper Role-playing Systems}

Dungeons \& Dragons is a fantasy role-playing game, which incorporates the \emph{d20 System}\cite{d20sys} developed by \emph{Wizards of the Coast}.
Its world consists of legendary creatures like elves, dwarves, dragons and other supernatural beings. The system is the first to be published commercially, it laid a very solid foundation that other systems have built upon, which makes it an interesting reference. The character creation relies on dice rolls to determine stats, and the player chooses an "alignment" which determines what kind of behaviour the character will exhibit. For example, a "chaotic evil" character will be thoroughly selfish, not caring for the lives and safety of others and solely focusing on achieving their goals.
The outcome of events and actions is decided with a dice pool containing various dice, from 4-sided up to 20-sided dice. For example, a character in a fight will roll dice to determine, if an attack hits the opponent, and how much damage is done. A player can also be asked to roll a die to check, if a stat or ability will affect a given situation. For example, an elf with very good hearing will be asked to roll dice to check, if they notice an ambush when travelling the roads at night.\cite{dnd}

\ac{wod} is a supernatural horror role-playing game, which incorporates the \emph{Storytelling System} developed by \emph{White Wolf Inc}. It exclusively uses 10-sided dice\cite{Appelcline2007}. Its world is inhabited by supernatural creatures like vampires, werewolves and ghosts. In character creation each player is assigned the same amount of points to spend on abilities, skills etc. There is no alignment guiding the player when making choices for their character, however the system does include merits and flaws. An example of a merit is ''Ambidextrous'', rendering the character able to use weapons with either hand without penalty. The system is unique in the way it handles character HP with regards to damage. \ac{wod} employs different categories of received damage, assigning a symbol to each type and having different healing rates for the types.\cite{wod} This means, that e.g. a stab wound has one type of damage, while being bashed on the head with a hammer is a different type of damage.

\subsection{Common characteristics}
\label{baseclasses}
The above mentioned systems have some basic attributes in common, which helps give an idea of what should be describable in the language.

Some important common characteristics are listed below:
\begin{itemize}
	\item The systems rely on character stats to calculate things such as damage, and even if they have different representations, all characters have some form of HP. They have modifiers for the probability of a successful outcome when performing actions, and things like abilities act as modifiers. It is important to be able to describe the characters' basic attributes, skills or spells with certain effects and resources such as magic, 
	\item The systems all employ turn-based battle, where the characters and \ac{npc}s take turns attacking each other. 
	\item The systems require the player to make choices each turn, either by "pre-programming" their characters or by actively choosing appropriate actions.
\end{itemize}

What is interesting here is the turn-based fight between characters in each system and this will be a focus in the project.
To help implement these turn-based battles, we take a look at the very successful computer role-playing game series ''Final Fantasy'' and get inspiration from their ruleset and implementation.

%Cyberpunk 2020 (Cyberpunk) is a role-playing game with a postmodern science-fiction setting, which incorporates a system called \emph{Interlock System} developed by \emph{R. Talsorian Games}. Its world consists mostly of humans, cyborgs and robots.

%The \emph{Interlock System} is skill-based instead of the traditional level-based system. Meaning that players get awarded points to spend on their skill sets instead of experience points. The outcome of events is decided with a similar dice pool to 'DnD, but using 10-sided dices to roll for success.\cite{cyberpunk}

\subsection{Computer role-playing game: ''Final Fantasy''}
The fight simulation will be based on the \ac{rpg} series \emph{Final Fantasy}. Final Fantasy is a fantasy role-playing game, released for the \emph{Nintendo Entertainment System} in 1987. The game was a huge success and has since then spawned a whole franchise named after the original title, which makes it an excellent reference when designing a combat system.
The Final Fantasy series have implemented various combat systems, both turn-based and real-time.\cite{ffantasy}

The focus of this project is on turn-based battle, to explain how it works here is an example:

Two opposing teams of characters and/or creatures are placed on a battlefield, a menu is presented to the player where they can choose an action to execute for each member of the player's team. After choosing an action for each member, the turn is run with each participating character/creature executing their action. The opposing team's actions are scripted by the developers and are therefore fully automated.
This is done until one of the teams' members are all out of combat.

Underlying are the calculations required to make use of all the various stats that define a character. In the first Final Fantasy games, characters have some basic stats such as; Strength, Agility and Intelligence. These attributes are used to calculate health, mana, attack power and defense to name a few, which then are used in damage calculations and ability/spell costs. The attributes and resources used in Final Fantasy I are listed below:

\textbf{Attributes}: Strength, Agility, Stamina, Magic, Attack, Defense, Evasion and Magic Defense.\\
\textbf{Resources}: HP (Health) and MP (Mana).


The gambit-system is a source of inspiration in this project, named after chess, it determines certain events and automated appropriate responses that can be set by the player, which allows for faster battles with larger parties of player-controlled characters. In other words the gambit system allows for a form of behaviour for characters.

\subsection*{Target group}
After carefully analysing the \ac{rpgs} and our desired implementation, the target group is defined as follows:

\ac{rpg} players and fans, especially within Pen \& Paper- and Computer- \ac{rpg} genres. There is no requirement for programming skills.
%RPG players, not programmers and therefore our choice of paradigm.
%Pen \& Paper games since they are easy to get into (rulebooks and stuff)
%source code to video games not easily aquired
\begin{comment}
\subsection{Summary}
As a proof of concept, the common characteristics have been chosen for constructing the rulebook, common characteristics being the before mentioned \emph{requirements} of the systems.
By using Characters classes, attributes, effects, abilities, resources and modifiers, a user can define a character in \langname{}, like it was a character sheet. If done correctly, this should make it easy to use for the target group.
\end{comment}
%sources: WoD - The World of Darkness (ISBN: 1-58846-484-9) by White Wolf Publishing
%sources: Cyberpunk 2020: The Roleplaying Game of the Dark Future (ISBN: 0-937-279-13-7)
%sources: Dungeons & Dragons Player?s Handbook: Core Rulebook I, v. 3.5  (ISBN: 0-7869-2886-7)
%sources: Gameplay of Final Fantasy (http://en.wikipedia.org/wiki/Gameplay_of_Final_Fantasy)
