\section{Analysis of RPG systems}
After analysing the question sheets sent to a part of the target-group we decided to take a closer look at \emph{WoD}(d10), \emph{DnD 3.5}(d20) and \emph{Cyberpunk}(d10). The reason for this, is to find common characteristics and try to adjust our language to the basic needs of these popular role-playing games.

\subsection*{Role-playing games}
The World Of Darkness (WoD) is a supernatural horror role-playing game, which incorporates a system called \emph{Storytelling System} developed by \emph{White Wolf Inc}.
Dungeons \& Dragons (DnD) is a fantasy role-playing game, which incorporates a system called \emph{d20 System} developed by \emph{Wizards of the Coast}.
Cyberpunk 2020 (Cyberpunk) is cyberpunk role-playing game, which incorporates a system called \emph{Interlock System} developed by \emph{R. Talsorian Games}.

\subsection*{Common characteristics}
The systems have some basic \emph{needs}, such as:
Character classes, attributes, effects, abilities, resources, modifiers and variation (dice rolls).
Their dice sizes and quantity differ, but are derived from the same concept: \textit{''To decide the outcome of a given task, you must roll a dice of a specific size, add modifiers, calculate the result and compare to the task's difficulty. If your result is higher, you succeed''}

What these systems have in common, and is implemented differently in each, is health points, damage and their calculations. This results in another approach to our engine?s damage calculations. While WoD and Cyberpunk have some very convenient representations for \emph{pen \& paper} role-play, they are not suitable for our engine.

\subsection*{Conclusion}
As a proof of concept, we choose the common characteristics of these systems for constructing the rulebook, which the engine will abide by. The Event system and fight simulation will be based on the older versions of the computer game \emph{Final Fantasy} (1-5) to simplify the implementation.

The Final Fantasy games have implemented various combat systems, turn-based and real-time.
Since our engine simulates battle and provides fairness through modifiers, we choose turn-based combat, where a battle is measured in turns, where a \emph{dexterity} or \emph{speed} modifiers can have effect on how many actions, a character is allowed each turn.
For simplicity we also choose to restrict the number of contestants to two.



%sources: WoD - The World of Darkness (ISBN: 1-58846-484-9)
%sources: Cyberpunk 2020: The Roleplaying Game of the Dark Future (ISBN: 0-937-279-13-7)
%sources: Dungeons & Dragons Player?s Handbook: Core Rulebook I, v. 3.5  (ISBN: 0-7869-2886-7)
%sources: Gameplay of Final Fantasy (http://en.wikipedia.org/wiki/Gameplay_of_Final_Fantasy)