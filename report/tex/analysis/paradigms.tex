\section{Language paradigms}
Within computer programming, a \emph{paradigm} is a model or framework for problem solving. Ambler et al \cite{paradigms1992} describe it thus:

\begin{quote}
A programming paradigm is a collection of conceptual patterns that together mold the design process and ultimately determine a program's structure.
\end{quote}

The granularity with which paradigms are defined varies somewhat between sources. While Ambler keeps to the understanding that a paradigm is a larger philosophy of solving problems computationally, other sources maintain that a paradigm can be considered on the level of techniques (Floyd on divide and conquer \cite{paradigms1978}). Within the context of this report, we constrain ourselves to the more abstract form as presented by Ambler in order to better represent the higher levels of paradigms 