\section{Language paradigms}
Within computer programming, a \emph{paradigm} is a model or framework for problem solving. \citeauthor{paradigms1992} describe it thus:

\begin{quote}
\emph{A programming paradigm is a collection of conceptual patterns that together mold the design process and ultimately determine a program's structure.}
\end{quote}

The granularity with which paradigms are defined varies somewhat between sources. While \citeauthor{paradigms1992} keeps to the understanding that a paradigm is a larger philosophy of solving problems computationally, other sources maintain that a paradigm can be considered on the level of techniques\cite{paradigms1978}. The report is constrained to the more abstract form as presented by the former in order to better represent the higher levels of paradigms. The point of the discussion is to present an overview of the major paradigms, their main points of interest and perceived the strengths/weaknesses of each. Following the convention of authors of both articles, a language that allows a paradigm to be used fully is referred to as a language that \emph{supports} the paradigm.

The discussion of paradigms is key in considering how the target users are expected to work with the designed language. By far most modern languages today do not constrain themselves to a single paradigm, instead supporting a variety of different approaches to better fit with the overall design of the language and its uses. As examples, $C++$ supports procedural, functional, object-oriented and generic paradigms, while $SQL$ supports both declarative and imperative approaches. In considering the target audience of the language, the paradigms that would ease their work while remaining effective are determined. The remainder of this section describes the major categories of paradigms (as described by Sebesta\cite{concepts_prog_lang}) with brief examples of underlying paradigms.

\subsection{Imperative}
The imperative paradigm is perhaps the one most naturally mapped to the actual architecture of modern computer systems. In it, an abstract storage model of the computer system is utilised, predominantly the Von Neumann architecture, wherein data and program are stored alongside one another in main memory. This leads to the use of variables as metaphors for memory cells to store data and some form of structured programming to encourage the stepwise execution of adjacent memory cells.

This basis in the architecture leads to several supporting paradigms, notably structured programming and object oriented programming.

\subsubsection*{Structured programming}
Programming within the imperative domain using only instruction jumps was a very daunting task. Code, with many branches of execution and repetitions, would wind up as proverbial ''spaghetti'' code as the only formal way of executing anything but the next instruction was the GOTO statement. A notable opponent of the GOTO statement was Edsger W. Dijkstra.\cite{goto}

Structured programming eased the implementation task by introducing notions of blocks, branching control structures and loops. While the final machine code will often still rely on the roots of procedural programming, these abstractions in higher programming languages vastly simplify the task of problem solving for programmers.

\subsubsection*{Object oriented}
Object oriented programming emerged with the advent of data-oriented programming. Where procedure-oriented methods focused on an algorithmic approach to problem solving (the necessary steps involved in creating the solution), data-oriented programming instead works with abstract data types that model the problem and solution.

The approach encapsulates procedures and variables within objects to create a more clear data model. In turn, this makes it a more natural task to model reality. Consider trying to describe the workings of a basic car using only procedures for managing components that can only be described atomically. Compare that to doing so with abstractions for wheels, chassis and the like, each with separate responsibilities and capabilities.

Furthermore the object oriented method introduced inheritance and polymorphism, extending the basic notion of classes of objects with similar representations to hierarchies of objects with shared ancestors while offering specialised implementations.

\subsection{Functional}
The functional paradigms focus on mimicking the notion of functions from mathematics as closely as possible. This results in shedding such concepts as states and iterations. Consider a mathematical function $f(x) = x \times 3+7$, where the value $x$ from the domain set is applied to the expression $x \times 3+7$, resulting in an element from the value set. In \emph{Scheme}, this function could be represented as

\begin{lstlisting}
(DEFINE (fun x)
    (+ 7 (* x 3))
)
\end{lstlisting}

and used with 

\begin{lstlisting}
(DEFINE val 3)
(fun val)
\end{lstlisting}

Note the use of named constants. Values are defined as constant expressions and passed as arguments (from the domain set) to a function that \emph{defines} the corresponding element in the value set. The function

\[
f(x) =
\begin{cases} \frac{x+3}{12} & x \texttt{ mod } 3 = 0 \texttt{ and } x \geq 9
\\
\frac{x}{12} & x = 12
\\
(x * 4)^2 & otherwise
\end{cases}
\]

Contains conditions and requires the use of recursive and conditional elements of the function languages. Again, in Scheme, this would look as follows:

\begin{lstlisting}
(DEFINE (^ x y)
    (IF (= y 0)
        1
        (* x (^ x (- y 1) ) )
    )
)

(DEFINE (fun x)
    (IF (>= x 9)
        (/ x 12)
        (IF (= x 12)
            (/ (+ x 3) 12)
            (^ (* x 4) 2 )
        )
    )
)           
\end{lstlisting}

The line breaks are only for readability and can be omitted. Note how every single token in the program code is either a list or atom (the two basic units of Scheme) - either an atomic value (a function name, a constant or a literal) or a list serving as a function call. Notice also how the integration of functions within functions work in exactly the same way as function composition does in mathematics. A final trait is how the lack of any form of variable or machine state completely eliminates side-effecting; the output will \emph{always} be the same for any given input.

\subsection{Logic}
Logic programming draws heavily from formal logic in predicate calculus, and indeed was inspired by the needs of the related mathematical disciplines. In an effort to create automated theorem proving, logic programming is - at its simplest - a question of stating a set of \emph{facts}, \emph{rules} and \emph{goals}. The first define the parameters of 

The logic programming paradigm makes use of Horn clauses to describe predicates and the facts that must hold true for the predicate to do so as well. Horn clauses are beyond the scope of this report, but for context a brief example is in order. using notation of boolean algebra, a horn clause could be written as 

\[
c_1 \cup c_2 \supset f
\]

and should be read as ''if $f$ holds true then one of either $c_1$ or $c_2$ holds true''. A program written in the logic programming style is comprised solely of clauses like this. The programmer must define sufficient facts (and a goal to be reached), and the compiler must be able to deduce the correct resolution of facts to reach this goal. The set of possible combinations that must be tested to find the solution very quickly becomes so large that algorithms used for solving the problem must be heavily optimised, and even small changes to the ordering of facts can markedly affect the performance of the program. While several algorithms with acceptable running times have been designed, it still remains a challenge for the programmer not only to supply sufficient facts to deduce a solution, but also in a manner that is not counter-productive to the overall running time of the program. While the latter can be argued to be a staple of any programming paradigm, none would have as great an impact as in the logic programming paradigm.

\subsection{Conclusion}
Considering the before mentioned paradigms, the one likely most approachable for beginners in the field of programming, is functional programming. The reason being that one is not required to tell the computer, step-by-step, how to calculate a result, but rather what is to be done. This paradigm is declarative in nature and in some cases easier to understand and learn.