\pagebreak
\section{Syntax theori}
% ~Hornbjerg
Every language, whenever it is English, German, or some other language, have some specified gramma, used for making sentence in the languages. The gramma of the languages is specified by rules, which desripe how to use the gramma corretly.   \\
Every sourcecode of a program, written in some programming language also consist of some sentence. Specify how a correct sentence in programming languages i written, the programming language uses a set of rules called the \textit{syntax} of the language. The syntax of a programming language can be descriped i deffirent ways, but in this section, we focus at three ways:
\begin{itemize}
\item{\ac{cfg}}
\item{\ac{bnf}}
\item{\ac{ebnf}}
\end{itemize}

\subsection{\ac{cfg}}
\ac{cfg} was first descriped in the middel of the 1950s by Noam Chromsky. A \ac{cfg} is a colloction of substitutions rules. The rules consist of two kinds of symbols; \textit{nonterminals} also know as \textit{variable} and \textit{terminals}. Each rule in a \ac{cfg} is formed as a line, which constist determents how substitut one (ore more) \textit{variables} with an other vaiable or terminal. \\
An example of a \ac{cfg} is 

\begin{tabular}{l l l}
$A$ & $\rightarrow$ & $0A1$ \\
$A$ & $\rightarrow$ & $B$ \\
$B$ & $\rightarrow$ & $\#$ \\
\end{tabular}

In the gramma above the rules shows that $A$ can be substituted by either $0A1$ or $B$. \\
The \ac{cfg} above will be called $G$, and the language which can be generated by the \ac{cfg} $G$ is called a \textit{context-free language} and is written $L(G)$, which means $L$ is generated by $G$.

\subsection{\ac{bnf}}
In 1959 a new formal notation used to specify the syntax of at programming language was introduced by John Backus, and the notation was later modified by Peter Naur, and the notation was called \ac{bnf}. \\
\ac{bnf} is an other way of descriping syntax of a programming language. Like a \ac{cfg}, a \ac{bnf} is a collection of rules, which also uses \textit{terminals} and \textit{nonterminals}. An example of \ac{bnf} used to descripe a simple Java assignment statement is written as

\begin{tabular}{l l l}
$<\texttt{assign}>$ & $\rightarrow$ & $<\texttt{var}> \; = \; <\texttt{expression}>$ 
\end{tabular}

where $<\texttt{assign}>$ is an abstract reprecentation of an assignment statement, and the rule specifies that $<\texttt{assign}>$ is defined as an instance of the abstraction $<\texttt{var}>$ followed by the `=' symbol, and then followed by the abstraction $<\texttt{expression}>$.

\subsection{\ac{ebnf}}
\ac{ebnf} is an extended version of \ac{bnf}. The \ac{bnf} has though time bin extended i several ways, and even though all the extenssions is not exactly the same, the are all called \ac{ebnf}. The extensions only makes the notation more readable and writeably, they does not enhance the descriptive power of \ac{bnf}. \\
One of the things included in one of the extensions of the \ac{bnf}, denotes an optional part at the right-side of the ``$\rightarrow$'', which is delimeted by brackets. for an example, an \textbf{\texttt{if-else}} statement, in the programming language C. The statement can be descriped

\begin{tabular}{l l l}
$< \texttt{if\_stmt}>$ & $\rightarrow$ $ \textbf{\texttt{if}} \; ( < \texttt{expression} > ) \;  < \texttt{statement} > \; [ \textbf{\texttt{else}} \; < \texttt{statement} > ]$
\end{tabular}

With use of \ac{bnf} the statement would be desriped \\

\begin{tabular}{l l l}
$< \texttt{if\_stmt}>$ & $\rightarrow$ &  $\textbf{\texttt{if}}  \; ( < \texttt{expression} > ) \;  < \texttt{statement} >$ \\
 & $|$ & $ \textbf{\texttt{if}} \; ( < \texttt{expression} > ) \;  < \texttt{statement} > \; \textbf{\texttt{else}} \; < \texttt{statement} > $
\end{tabular}