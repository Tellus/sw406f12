\chapter{Evaluation}
%Evaluate evaluate ! Don't worry just keep testing...
In this chapter the different phases of the project is evaluated, along with an overall evaluation of the project process.

\section{Analysis}
The analysis was started too late and ended too early. The problem statement and the initiating problem was not properly defined to start with. This made the report lose focus and a lot of effort was wasted on things that were later scrapped. As an example, the target group was redefined two times, which meant that analytical work related to earlier target groups had to be redone.
The choice of paradigm was never properly discussed until the design phase, which ended up being quite the hindrance to good design.

\section{Design}
The design of the language was started too early in the process, but too late with regards to time. The analysis was not done by the time design started, and this meant that some syntax was suddenly deprecated and some was plain wrong when considering the target group. The choice of paradigm also opened up problems, seeing as the group only had experience with object-oriented and imperative paradigms. The paradigm also did not fully support what the group initially wished to describe in the language. As an example the group originally wanted users to be able to describe rule sets in the language, which would require the use of imperative statements that should not be accessible in a declarative paradigm. The group realised that this was the wrong paradigm so late in the process that they felt it would be counter-productive to change it, and instead workaround solutions were employed to have the syntax fit in the proper paradigm instead of simply writing it the easiest way.

The final design, however, closely resembles what the target group is used to seeing in pen and paper roleplaying when creating characters, which should make it easy to learn and use. Definition of behaviour, which does not occur on character sheets, is simple and straightforward, and the overall design, though limited in comparison to the initial idea, turned out fine.

\section{Implementation}
The implementation was the most rushed part of the project in total. The process was started so late that the group at one point was even worried it might not be done in time. This led to code which is more inefficient and slow than it could be, but all the desired functionality is implemented which was a success to the group.

The group chose to implement every phase of the compiler by hand, which gave a good understanding of how the lexer and parser works, so the group considers this a good idea.

\section{Overall evaluation}

All in all, the group feels that the project could have had better project planning and a better distribution of project roles. Seeing as the group work was marked by too few people, having two members departing early in the process, the workload was heavier than it could have been. Despite poor project planning and heavy workload the team worked hard and still achieved their goals.

The group felt that the contract written at the start of the semster was very good, but the upholding of the group contract was lacking, especially nearing the end of the semester. They received excellent supervision along the way, and the weekly meetings with the supervisor was a great motivator to get things done.

\section{Future Work}
Given the time constraints of a university semester, there are features that have not been implemented. Below is a list of improvements and additions that would help make the project even better:
\begin{itemize}
\item The engine should be modified to allow an external file with a ruleset to give the possibility of simulating battles in various systems. 
\item The compiler should be updated to have better type checking as there are currently some false positives.
\item The language should be extended to include all aspects of \ac{rpg}s such as Items and support for continuous effects, new Attributes and Resources and inheritance.
\item The compiler should support declaration at any given point in the code whereas it currently only supports declaration before use.
\item The language could be redesigned to fit the imeperative paradigm, which would fix many problems the group has had during the project.
\end{itemize}