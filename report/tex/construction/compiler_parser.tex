\subsection{Parser}
The implemented parser is a recursive descent, LL-1 parser, which converts the stream of tokens to an \ac{ast} based on the grammar of \langname{}. It takes the stream generated by the scanner and outputs the tree.

The nodes of the \ac{ast} are quite alike, so only the abstract syntax node will be described. Every node of course has a type for type checking. Besides this, it also has a boolean value to determine if the node represents dynamic or static data, a scope reference and visit function used in type checking, and functions to print the tree. It also has emit functions used in code generation.
 
The parser class, of course, has a parse function, an input function and accept-functions for the different types of nodes the tree should accept.

The parser starts by checking, if the first node encountered is a token of type "Make", which marks the beginning of a program. It creates the appropriate node and then recursively accepts correct derivations from this rule, i.e. it accepts declarations, then assignments etc. according the the context free grammar. The accepted nodes are added as children to the tree. An advance function allows the parser to proceed to the next token after accepting the current. This function has a built-in type check: it performs a bitwise-and operation on the input token and the expected token, and if the result is the input, i.e. the two are the same, it performs the advance. Should the two be different, however, the function returns an error to inform the user that an unexpected token was encountered.

\begin{lstlisting}[language = c++]
trees::AbstractSyntaxNode *Parser::accept_expression()
{
	trees::ArithmeticNode *node = new trees::ArithmeticNode();

	trees::AbstractSyntaxNode *lhs = this->accept_term();

	FORMAT_TOKENTYPE op = this->input();
	if (op != tokens::TokentypePlus && op != tokens::TokentypeMinus)
	{
		return lhs;
	}

	node->add_child(lhs);
	node->op = this->advance()->value;
	node->add_child(this->accept_expression());

	return node;
}
\end{lstlisting}

As seen above the accept-functions return an \ac{ast}-node and accept no arguments. In this case, the function should accept an expression, and so a new node of the type arithmetic node is created and an abstract syntax node is instantiated to encompass the left-hand side, and the function accept\_term() is run on this. 

The function checks whether any operators follow the expression, and if none do the left-hand side is returned, so that no empty nodes are created for operators that are not present.

Fianlly the function adds the appropriate children of the node, and calls itself on a child-node to check for additional expressions.
