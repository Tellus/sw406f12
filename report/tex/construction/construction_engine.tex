\section{Engine}
While the \langname{} compiler chain parses, validates and translates program code, the result cannot be run on its own. The compiler must be complemented by a run-time library, an interpreter or perhaps some form of virtual machine that will support the execution of a \langname{}-defined battle scenario. A run-time environment has been developed in parallel with the compiler, enabling coded scenarios to be resolved as described by the turn-based semantics. This section will describe the inner workings of the engine, its basic design, architecture and form.

\begin{figure}
\includegraphics[scale=1]{img/engine_class_design}
\caption{The abstract class design of the engine.}
\end{figure}

\subsection{Base philosophy}
The stated intention of the engine is to simulate a scenario according to the definitions set up by the \langname{} code. This results in the need for several features, both for convenience and correctness.

\subsubsection{Data types}
The seven basic data types of \langname{} should be mapped to similarly functioning classes within the engine's framework. The compiler's code generation stage - in particular, the templates it utilises - are vastly simplified by defining them in the engine. This has the added benefit of allowing us to better define the engine before any \langname{} code has even been written. The end result is effective delegation of responsibilities. The code generator should by no means be weighed down by lengthy class implementations, when it should instead make use of small, efficient templates.